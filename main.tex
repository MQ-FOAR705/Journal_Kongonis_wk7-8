\documentclass{article}
\usepackage[utf8]{inputenc}

\title{Learning Journal wk7 and wk8}
\author{Thomas Kongonis}
\date{September-October 2019}

\begin{document}

\maketitle

\tableofcontents

\section{Open Refine}

\subsection{Lesson 2: Working With Open Refine}

\begin{itemize}
\item{\textbf{Exercises}}

\begin{enumerate}

\item{\textbf{2/10/19 14:32 Intention:} Complete Exercise 'Using Facets 1'.}
\subitem{\textbf{Result:} There is definitely a lot of errors in this. The Chirdozo with one entry is probably an error, same with Ruca. This is also the same for the Ruaca-Nhamuenda. Finally, the number 49 is clearly an error that cant be solved at this stage. I would diagnose the incorrect one as the entry that has the lowest frquency.}

\item{\textbf{Intention:} Complete Exercise 'Using Facets 2'}
\subitem{\textbf{Result:} There are 19 inputs. It appears as if it has been formatted by text. After using the transform function to format them as dates, it would appear that the majority of interviews were done in 2016.}

\item{\textbf{Intention:} Complete Exercise 'Transforming Data 1'.}
\subitem{\textbf{Result:} Implemented the codes to remove the commas and brackets as instructed. Was a success although initially i missed some quotation marks and there was an error.}

\item{\textbf{Intention:} Complete Exercise 'Transforming Data 2'.}
\subitem{\textbf{Result:} Implemented facet, mobile phone and radio were both at 86 and the lowest 2 were car at 3 and computer at 2.}

\item{\textbf{Intention:} Complete Exercise 'Transforming Data 3'.}
\subitem{\textbf{Result:} Implemented each code separately, ended up with a lot of results grouped together. It showed that November had the highest frequency but it was combined with October.}

\item{\textbf{Intention:} Complete Exercise 'Transforming Data 4'.}
\subitem{\textbf{Result:} Implemented cleaning code to the cells suggested and received no error messages.}

\item{\textbf{Intention:} Complete Exercise 'Using Undo and Redo'.}
\subitem{\textbf{Result:} Went into the undo redo section and individually went though each step as instructed. It seems like a really useful way to see mistakes.}

\end{enumerate}
\end{itemize}



\subsection{Lesson 3: Filtering and Sorting with Open Refine}

\begin{itemize}
\item{\textbf{Exercises}}

\begin{enumerate}

\item{\textbf{Intention:} Complete Exercise 'Filtering'.}
\subitem{\textbf{Result:} Completed the instructions. In the facet it just showed the inputs that included mabat.}

\item{\textbf{Intention:} Complete Exercise 'Excluding Entries'}
\subitem{\textbf{Result:} By using the include and exclude functions, i am able to only showcase entries i want to see.}

\item{\textbf{Intention:} Complete Exercise 'Sort'.}
\subitem{\textbf{Result:} Implemented the sort code as instructed and shown in class. As discussed, the 0 result would be a mistake.}

\item{\textbf{Intention:} Complete Exercise 'Sorting by Multiple Columns'.}
\subitem{\textbf{Result:} When we compare the gps coordinates of 49, we can see that its clustered with chodrizo village. Therefore we can infer that it is also chodrizo.}



\end{enumerate}
\end{itemize}



\subsection{Lesson 4: Examining Numbers in Open Refine}

\begin{itemize}
\item{\textbf{Exercises}}

\begin{enumerate}

\item{\textbf{Intention:} Complete Exercise 'Numbers'.}
\subitem{\textbf{Result:} Changed the suggested cells to numbers and it worked. Did it for village and it changed the 49 to a green number as i forgot to change it to chodrizo.}

\item{\textbf{Intention:} Complete Exercise 'Numeric Facet'.}
\subitem{\textbf{Result:} Changed the top cell in no members to abc then did the numerical facet and got a non numerical input.}



\end{enumerate}
\end{itemize}





\subsection{Lesson 5: Using Scripts}

\begin{itemize}
    \item{textbf{Exercise: Saving Work as Script.}}
    
    \subitem{\textbf{Result:} Work successfully saved as .Json script.}
\end{itemize}



\subsection{Lesson 6: Exporting and Saving Data from Open Refine}

\begin{itemize}
\item{\textbf{Exercises}}

\begin{enumerate}

\item{\textbf{Intention:} Complete Exercise 'Saving and Exporting a Project'.}
\subitem{\textbf{Result:} Exported the .tar file, unpacked it and saw a history folder full of changes. There was also a data compressed file. Then i exported a csv file.}


\end{enumerate}
\end{itemize}




\subsection{Lesson 7: Other Resources in Open Refine}

\begin{itemize}
\item{\textbf{Exercises}}

\begin{enumerate}

\item{\textbf{Intention:} Complete Exercise 'Using Online Resources to get Help with Open Refine'.}
\subitem{\textbf{Result:} Went on sites and had a look.}


\end{enumerate}


\subsection{concluding comments}
\begin{itemize}
\item{\textbf{Response to Openrefine} In comparison to the Shell command, open refine was significantly easier and took about one tenth of the time to complete the work. I can most certainly see the utility of such a program and it has the potential to be useful to me in the future if i find myself needing to clean data.} 
\end{itemize}
\end{itemize}

\section{Project}
\subsection{Gathering Translations}

\begin{itemize}

\item{\textbf{ 30/9/2019- Public Domain:} Searching for the translations i would be using i had one key rule in mind. This rule was that each translation i decided to utilise was irrefutably seen as public domain and was noted as such. The last thing we want is litigation!}

\item{\textbf{1/10/2019- 5 Translations:} From this search, i have found and downloaded 5 of these public domain translations.}

\item{\textbf{1/10/2019- Formatting:} All of these translations were in different format. Some were .txt, some were coded in the html code of a site, others were pdf. As such, i have begun the process of transferring these all into .txt format. Part of this project will now be an attempt to utilise a shell code to tag and edit these documents en masse and i will also be using the shell command for any directories and document transfers for the sake of showcasing my command over the shell command.}
 
\end{itemize}

\subsection{Deciding on a direction}
\begin{itemize}
\item{\textbf{26/9/2019- Utilising shell script} After trying to figure out the best and easiest way to get my desired result, i settled upon thee utilisation of a shell script and the usage of wildcards to pull sections of documents from a specified directory. This will get the desired result with minimal work and also allows for greater effort to be put into making the process as seamless as possible and utilising the shell command for the entire process.}

\item{\textbf{26/9/2019- Process:} As previously mentioned, i will be documenting each code and utilising the shell command for completing the entirety of my proof of concept. This will include the creation of directories, moving documents, creating documents, implementing my user stories and all things associated with my proof of concept. My desire for this project is that from start to finish, that i will be able to utilise the shell command completely in lieu of a gui unless drastically required.  }
\end{itemize}

\section{Journal and Latex}

\subsection{Creating journal template}
\begin{itemize}
 \item{\textbf{25/9/2019- Creating Template} A significant point of difficulty that i had with the previous journal was the time that it took for me to create the latex document with formatting that i liked. To remedy this, i decided that i would create a few lines of code within a template that i would be able to utilise to create basic templates for this journal and the final journal. I estimate that doing this has cut the time needed to write this jouurnal in half.}
 
\end{itemize}

\end{document}
